	% 	a. Template
	% 	b. Engine
	% 	c. Template Engine
	% 	d. History of Templating
\chapter{Was ist Web 2.0?}

\section{Definition und Vergleich}
Grundlegend versteht man unter Web 2.0 eine veränderte Form und Nutzung des Internets, welches aber keine neue Technologie ist, sondern ein Nachfolger des Web 1.0. 
Beim Vorgänger Web 1.0 stand die Verbreitung von Wissensgut und Informationen sowie der Produktverkauf durch Websitebetreiber im Vordergrund.\cite{o'reilly} Das Hauptaugenmerkmal lag auf der Kommunikation zwischen Anbieter und Verbraucher, sodass die Mehrheit der Nutzer des Web 1.0 überwiegend Konsumenten waren.


\section{Nutzergenerierte Inhalte}
Das Web 2.0 ermöglicht dem Nutzer selbst Inhalte und Daten ohne vertieftes technisches Wissen bereitzustellen. Eine vereinfachte Benutzeroberfläche befähigt den User seine Webinhalte zu veröffentlichen und anderen Internetnutzern Informationen zur Verfügung zu stellen.
Durch leicht zugängliche Wikis und Blogs kann der Prosument selbst Beiträge 
leisten und trägt somit dazu bei, wesentlich höhere Daten- und Austauschraten zu erzielen. \cite{hemken}
Über Social Networks und Foren kann der Benutzer nun leichter und intensiver mit anderen über das Internet interaktiv kommunizieren und sich mitteilen. \cite{itwissen}
Durch das Web 2.0 hat sich das Nutzerverhalten enorm geändert, da der User nun nicht mehr ausschließlich Konsument der Webinhalte ist, sondern diese auch aktiv mitgestaltet und veröffentlichen kann.