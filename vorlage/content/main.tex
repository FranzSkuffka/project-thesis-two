\chapter{Einleitung}
Ziel dieser Projektarbeit ist die Entwicklung eines Systems, dass es dem Nutzer erlaubt, Information ohne Eingabemaske einzupflegen. Dabei soll die Eingabe automatisch, wenn möglich, einer Klasse in der Problemdomäne CRM zugeordnet werden. Wir betrachten Texte, die nur bedingt prosaisch sind und deshalb nur wenig grammatikalische Strukturen aufweisen. In diesem Fall müssen andere Merkmale zur Verarbeitung der Eingabe berücksichtigt werden.
Zu Beginn des ersten Teils der Arbeit werden die konkreten funktionalen Anforderungen an das System definiert. Als Nächstes betrachten wir die theoretischen Prozessschritte, die vom Rohtext zum eingeordneten und extrahierten Datensatz führen. Der zweite Teil behandelt die Schwierigkeiten die in Anbetracht der Qualität der Eingabedaten und der technischen Mittel aufkommen sowie diesen entgegengewirkt werden kann. Im Folgenden werden aus den theoretischen Überlegungen abgeleitete systemarchitektonische Strukturen auf die tatsächliche Umsetzung übertragen.
Zuletzt wird der Erfolg des Systems Anhand der Anforderungen ermittelt und ein Ausblick auf praktische Einsatzmöglichkeiten geboten.

non-tech affinity \cite{hemken}
human indexing is expensive, cost reduction because expert-labour
learning by doing encouraged. people get introduced to the problem domain throught the suggestions served by the system.
Ziel: Klassifikation und Extraktion von Texten
domäne crm, ausgewählte kategorien
\chapter{Hauptteil}
\section{Funktionale Anforderungen}
\subsection{Interaktionsmodell}
definition setting,user scenario,interaction model
system erhält feedback von nutzer - trainierbar
interaktionsmodell bzw. lernprozess - learning when something was missed if user enters it and it is also found inside the text
teilautomatische "interaktive" klassifikation und extraction, weil: trainingsdaten rar? und vertrauen auf den classifier gering.
\subsection{Datenverarbeitung}
extensible: insert own types with certain features?
echte anforderung: email-footer
fokus auf texte die einer klasse entsprechen
automatisierte generierung von metadaten, metadaten werden hier das neue zentrum der interessen
exogenous information - phone note for example on ingoing phone call
a priori?
Simon Mayer becomes Simon Mayer (recognised kind: account)

"Strategy" becomes Strategy timeline with numbers displayed
\section{Theoretische Herangehensweise}
Texte fast nie grammatikalisch korrekt
endogene Information
problem of notes - bad spelling
\subsection{Klassifikation der Eingabe}
Categorization - using chrome? for email footer for example
Occam's Razor?
characterization - finding features of certain category
characteration through template?
\subsection{Extraktion von Information}

\section{Implementierung}
\subsection{Werkzeuge}
alchemy api
learning classifiers
regular expressions

\subsection{Trainingsdatensatz}
- endogenous knowledge only in paper notes...
- nonoverlapping categories but multiple categories in one note. -> separation
\subsection{Implementation des Klassifikationsmechanismus}
\subsection{Implementation des Extraktionsmechanismus}
\section{Evaluierung des Systems}
\subsection{Prüfung der funktionalen Anforderungen}
\subsection{Präzision}


\chapter{Schluss}
\section{Zusammenfassung}
\section{Fazit}
\section{Ausblick}

\chapter{Notizen}

section{To Do}
Jones fragen
binäre entscheidung? oder single-label?
find thesaurus for crm
binäre entscheidung? oder single-label?
expert interview - ivan kravchenko?
recognizing special cases (what did i mean here?????)

Datensätze:
E-Mail-Footer
Exporte der CRM-Datensätze

section{Literatur}
Knowledge‐enabled customer relationship management: integrating customer relationship management and knowledge management concepts

"An Adaptive User Interface Based On Personalized Learning" pdf
