% arara: xelatex: {options: -output-directory=./output}
% Hinweis: Optionen der Dokumentenklasse werden an alle folgenden \usepackage{package} Befehle weitergegeben
\documentclass[
	fontsize=12pt,
	paper=a4,
	parskip=half,
	twoside=false,
	numbers=noenddot,	% Kein Punkt am Ende einer Überschrift
	%draft=true,			% Deckt Schwächen auf: overfull und full boxes werden markiert; Bilder werden nicht geladen
	bibliography=totoc,	% Literaturverzeichnis ins Inhaltsverzeichnis aufnehmen
	listof=totoc,		% Tabellen- und Abbildungsverzeichnis ins Inhaltsverzeichnis aufnehmen
	titlepage=true,		% Separate Titelseite; Gestaltung mit Hilfe der Titlepage-Umgebung
	headsepline=true,	% Kopflinie aktivieren
	footsepline=true,	% Fußlinie aktivieren
	abstracton			% Abstract aktivieren
]{scrreprt}

% Zeichenkodierung Ausgabe ist T1-Kodierung: Wichtig für die Ausgabe von Umlauten
\usepackage[T1]{fontenc}

% Schrift festlegen
\usepackage{fontspec}

\setmainfont[
BoldFont=Arial Bold.ttf,
ItalicFont=Arial Italic.ttf,
BoldItalicFont=Arial Bold Italic.ttf
]{Arial.ttf}

\usepackage{titlesec}
\renewcommand{\sectfont}{\rmfamily}
% Sprachauswahl für Lokalisierungen und Silbentrennung
\usepackage[german]{babel}

% Zitate: Anführungszeichen automatisch anhand der Sprache wählen
\usepackage[babel=true]{csquotes}

% Source-Code-Listings
\usepackage{listings}
\lstset{
    language=SQL,
    basicstyle=\ttfamily
} 

% BibTeX-Symbol
\usepackage{texnames}

%Farbpaket laden
\usepackage{xcolor}

% Symbole, z.B. Haken
\usepackage{pifont}

% Zeilen in Tabellen zusammenfassen
\usepackage{multirow}

% Silbentrennung kann bei bestimmten Wörten mit Hilfe von diesem Paket deaktiviert werden 
\usepackage{hyphenat}

% Abkürzungsverzeichnis
\usepackage[printonlyused, withpage]{acronym}
% Abstand mit Punkten füllen

% Tiefe des Inhaltsverzeichnisses
\setcounter{tocdepth}{2}

% Punkte im Inhaltsverzeichnis
\usepackage{tocstyle}
\usetocstyle{allwithdot}

% Zum Einbinden von PDF-Dateien.
\usepackage{pdfpages}

% Paket zum Anpassen von Kopf- und Fußzeilen
\usepackage[plainfootsepline, plainheadsepline, headsepline, footsepline, automark]{scrpage2}
\setlength{\headheight}{1.1\baselineskip}
% Liniendicke
\setheadsepline{0.1pt}
\setfootsepline{0.1pt}

% Kopf- und Fusszeile löschen
\clearscrheadfoot
% Kopf- und Fusszeile aktivieren
\pagestyle{scrheadings}

% Kopf links
\ihead[\titel]{\titel}

% Fuss links
\ifoot[\verfasser]{\verfasser}
% Fuss rechts
\ofoot[\pagemark]{\pagemark}

% Grafiken einbinden
\usepackage{graphicx}
% Pfad zu den Grafiken
\graphicspath{{imagery/}}

% Seitenränder setzen
\usepackage[left=3.5cm, right=2.5cm, top=2.5cm, bottom=3cm]{geometry}
\renewcommand*\chapterheadstartvskip{\vspace*{0cm}}

% Zeilenabstand auf 1.5 setzen
\usepackage{setspace}
\onehalfspacing

% Literaturverzeichnis
%\usepackage[backend=bibtex,style=authoryear]{biblatex}
\usepackage[backend=bibtex,style=alphabetic]{biblatex}
\bibliography{./literature/Literature.bib}
\renewcommand{\bibname}{Literaturverzeichnis}
%Referenz zu URLs
\usepackage{url}

% Glossar
\usepackage[acronym,toc,nonumberlist]{glossaries}
\makeglossaries

% Titel als Referenzierung verwenden
\usepackage{titleref}

% Währungen
\usepackage{textcomp}

% Fussnoten fortlaufend nummerieren.
\usepackage{chngcntr}
\counterwithout{footnote}{chapter}

% Persönliche Daten
\newcommand{\titel}{Implementierung eines Systems zur Klassifikation von und Extraktion von Information aus Texten der Domäne CRM}
\newcommand{\art}{Projektarbeit des zweiten Studienjahres}
\newcommand{\studienbereich}{Wirtschaft}
\newcommand{\studiengang}{Onlinemedien}
\newcommand{\verfasser}{Jan Wirth}
\newcommand{\kurs}{ON13}
\newcommand{\ausbildungsbetrieb}{visual4 GmbH}
\newcommand{\betreuer}{Prof. Dr. Arnulf Mester}
\newcommand{\abgabedatum}{}
\newcommand{\unterschrift}{\rule{5cm}{0.2pt}}

% Links- und PDF-Einstellungen
\usepackage[hidelinks]{hyperref}
\hypersetup{
	pdfauthor = {\verfasser},
	pdftitle = {\titel},
	pdfsubject = {\art},
	pdfkeywords = {},
	pdfstartview = {Fit},
	colorlinks = {false},
	breaklinks = {true},
	bookmarksopen = {true}
}

% Verhinderung von Schusterjunge und Hurenkind
\clubpenalty = 10000
\widowpenalty = 10000
\displaywidowpenalty = 10000

% Seitenzäler für große, römische Zahlen
\newcounter{RomanPagenumber}

% Abkürzungen
\newcommand{\dash}{d.\,h.}
\newcommand{\zB}{z.\,B.}

% Zitate in neue Zeile rücken
\usepackage{breakcites}

% Glossary capitalization
\usepackage{mfirstuc}
% \renewcommand{\glsnamefont}[2][]{\capitalisewords{#1}\xspace#2}


\begin{document}
	% Title page
	\thispagestyle{plain}

\begin{titlepage}
	\enlargethispage{4.0cm}

	\begin{center}		
		\raisebox{-.5\height}{\includegraphics[width=3cm]{imagery/visual4}}
		\hfill
		\raisebox{-.5\height}{\includegraphics[width=6cm]{imagery/dhbw_mosbach}}
		\\
		\vspace{3cm}
		\large{\textbf{Fachbereich: \studienbereich}}\\
		\vspace{0.5cm}
		\large{\textbf{Studiengang: \studiengang}}\\
		\vspace{1.0cm}
		\Large{\textsc{\textbf{\titel}}}\\
		\vspace{1.0cm}
		\large{\textbf{\art}}\\
		\vspace{1cm}
		\vspace{1cm}
		\vspace{3cm}
		\begin{tabular}{rl}
			Autor:					& \verfasser\\
			Kurs: 					& \kurs\\ 
			Ausbildungsbetrieb:		& \ausbildungsbetrieb\\ 
            Wiss. Ansprechpartner:  & \betreuer\\
			Abgabedatum:		& \abgabedatum\\
			\vspace{0.5cm}\\
			Unterschrift:				& \unterschrift\\
		\end{tabular} 
	\end{center}
\end{titlepage}

	
	% Ab hier große, römische Seitenzahlen
	\pagenumbering{Roman}
	
	%falls nicht der Titel der Arbeit oben stehen soll, bitte einkommentieren
	%\ihead[\headmark]{\headmark}
	
	% Sperrvermerk
	%\include{content/Sperrvermerk}
	
	% Abstract
	%\include{content/Abstract}
	
	% Eidesstattliche Erklärung
	\include{content/Erklaerung}
	
	% Contentssverzeichnis
	\tableofcontents
	\pagebreak
	
	% Abbildungsverzeichnis
	%\listoffigures
	%\pagebreak
    
    % Code Listing Verzeichnis
    %\lstlistoflistings
    %\pagebreak
	
	% Tabellenverzeichnis
	%\listoftables
	%\pagebreak
	
	% Abkürzungsverzeichnis
	%\addchap{List of Abbreviations}
	%\input{content/Abbreviations}
	%\pagebreak
	
	% Abschnitt beenden
	\clearpage
	
	% Seitenzähler erhält den Wert der aktuellen groß römisch nummerierten Seite; Zwischenspeichern für später
	\setcounter{RomanPagenumber}{\value{page}}
	
	% Ab hier arabische Seitenzahlen
	\pagenumbering{arabic}
		
	% Kopf links auf aktuelles Kapitel ändern
	\ihead[\headmark]{\headmark}

% 1. Introduction
%   1.1 Motivation
%   1.2 Definition
%     a. Template
%     b. Engine
%     c. Template Engine
% 2. Demand
%   2.1 Usage Scenarios
%     a. XSJS Functions
%     b. Stored Procedures
%     c. Database Triggers
% 3. Architecture & Environment
%   3.1 Template
%     a. Token Principle
%     b. Meta Code
%   3.2 Template Engine
%     a. HANA XS
%     b. Generic Architecture
%     c. Required Classes & Entities
% (4. Implementation)
% 4. Conclusion

	
	\chapter{Einleitung}
Automatisieren ist immer günstiger
\section{Zielgruppe}
UX Designer, Softwareentwickler
Grundwissen; CRM, Softwareentwicklung, Datenbanken, 
\section{Problemstellung}
Wachsende Komplexität
Einfachere Interfaces
Eingabeformulare als Barriere
CRM-Software möglicherweise Hinterher
Efforts von Salesforce

\section{Motivation}
human indexing is expensive, cost reduction because expert-labour \cite{shneiderman}
non-tech affinity 
bei großen system fällt Überblick schwer.

	\clearpage
	
	% Ab hier weiter mit großen, römischen Seitenzahlen
	\pagenumbering{Roman}
	
	% Seitenzähler auf zwischengespeicherten Wert für große, römische Zahlen setzen
	\setcounter{page}{\theRomanPagenumber}

	% Kopf links auf Titel ändern
	\ihead[\titel]{\titel}	
	
	% Anhang
	%\addchap{Appendices}
	%\appendix
	%\input{content/Anhang}
	
	% Beigaben
	%\include{content/Beigaben}
	
	% Glossar, falls benötigt
	%\newacronym{js}{JS}{JavaScript}
\newacronym{ie}{IE}{Informationsextraktion}
\newacronym{ae}{AE}{automatische Entitätserkennung}
\newacronym{crm}{CRM}{Costumer Relationship Management}
\newacronym{b2c}{B2C}{Business to Customer}
\newacronym{b2b}{B2B}{Business to Business}

% Entität
% Objekt
% Token


	%\makeglossaries
	%\printglossary
	
	% Literaturverzeichnis
	\printbibliography
	
	% Aktuellen Abschnitt beenden
	\clearpage
\end{document}
