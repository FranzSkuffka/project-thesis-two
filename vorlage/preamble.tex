% Hinweis: Optionen der Dokumentenklasse werden an alle folgenden \usepackage{package} Befehle weitergegeben
\documentclass[
	fontsize=12pt,
	paper=a4,
	parskip=half,
	twoside=false,
	numbers=noenddot,	% Kein Punkt am Ende einer Überschrift
	%draft=true,			% Deckt Schwächen auf: overfull und full boxes werden markiert; Bilder werden nicht geladen
	bibliography=totoc,	% Literaturverzeichnis ins Inhaltsverzeichnis aufnehmen
	listof=totoc,		% Tabellen- und Abbildungsverzeichnis ins Inhaltsverzeichnis aufnehmen
	titlepage=true,		% Separate Titelseite; Gestaltung mit Hilfe der Titlepage-Umgebung
	headsepline=true,	% Kopflinie aktivieren
	footsepline=true,	% Fußlinie aktivieren
	abstracton			% Abstract aktivieren
]{scrreprt}

% Zeichenkodierung Input ist UTF-8: Umlaute können direkt eingegeben werden
\usepackage[utf8]{inputenc}

% Zeichenkodierung Ausgabe ist T1-Kodierung: Wichtig für die Ausgabe von Umlauten
\usepackage[T1]{fontenc}

% Schrift festlegen
\usepackage{fontspec}

\setmainfont[
BoldFont=Arial Bold.ttf,
ItalicFont=Arial Italic.ttf,
BoldItalicFont=Arial Bold Italic.ttf
]{Arial.ttf}

\usepackage{titlesec}
\renewcommand{\sectfont}{\rmfamily}
% Sprachauswahl für Lokalisierungen und Silbentrennung
\usepackage[german]{babel}

% Zitate: Anführungszeichen automatisch anhand der Sprache wählen
\usepackage[babel=true]{csquotes}

% Source-Code-Listings
\usepackage{listings}
\lstset{
    language=SQL,
    basicstyle=\ttfamily
} 

% BibTeX-Symbol
\usepackage{texnames}

%Farbpaket laden
\usepackage{xcolor}

% Symbole, z.B. Haken
\usepackage{pifont}

% Zeilen in Tabellen zusammenfassen
\usepackage{multirow}

% Silbentrennung kann bei bestimmten Wörten mit Hilfe von diesem Paket deaktiviert werden 
\usepackage{hyphenat}

% Abkürzungsverzeichnis
\usepackage[printonlyused, withpage]{acronym}
% Abstand mit Punkten füllen
\renewcommand*\bflabel[1]{\textbf{\normalsize{#1}}\hfill}

% Tiefe des Inhaltsverzeichnisses
\setcounter{tocdepth}{2}

% Punkte im Inhaltsverzeichnis
\usepackage{tocstyle}
\usetocstyle{allwithdot}

% Zum Einbinden von PDF-Dateien.
\usepackage{pdfpages}

% Paket zum Anpassen von Kopf- und Fußzeilen
\usepackage[plainfootsepline, plainheadsepline, headsepline, footsepline, automark]{scrpage2}
% Liniendicke
\setheadsepline{0.1pt}
\setfootsepline{0.1pt}

% Kopf- und Fusszeile löschen
\clearscrheadfoot
% Kopf- und Fusszeile aktivieren
\pagestyle{scrheadings}

% Kopf links
\ihead[\titel]{\titel}

% Fuss links
\ifoot[\verfasser]{\verfasser}
% Fuss rechts
\ofoot[\pagemark]{\pagemark}

% Grafiken einbinden
\usepackage{graphicx}
% Pfad zu den Grafiken
\graphicspath{{imagery/}}

% Seitenränder setzen
\usepackage[left=3.5cm, right=2.5cm, top=2.5cm, bottom=3cm]{geometry}
\renewcommand*\chapterheadstartvskip{\vspace*{0cm}}

% Zeilenabstand auf 1.5 setzen
\usepackage{setspace}
\onehalfspacing

% Literaturverzeichnis
%\usepackage[backend=bibtex,style=authoryear]{biblatex}
\usepackage[backend=bibtex,style=alphabetic]{biblatex}
\bibliography{literature/Literature.bib}
\renewcommand{\bibname}{Literaturverzeichnis}
%Referenz zu URLs
\usepackage{url}

% Glossar
\usepackage[toc,nonumberlist]{glossaries}

% Titel als Referenzierung verwenden
\usepackage{titleref}

% Währungen
\usepackage{textcomp}

% Fussnoten fortlaufend nummerieren.
\usepackage{chngcntr}
\counterwithout{footnote}{chapter}

% Persönliche Daten
\newcommand{\titel}{Was ist Web 2.0?}
\newcommand{\art}{Übung zum wissenschaftlichen Arbeiten}
\newcommand{\studienbereich}{Wirtschaft}
\newcommand{\studiengang}{Onlinemedien}
\newcommand{\verfasser}{Jill Zander}
\newcommand{\kurs}{ON13}
\newcommand{\ausbildungsbetrieb}{HITS gGmbH}
\newcommand{\abgabedatum}{02.03.2014}
\newcommand{\unterschrift}{\rule{5cm}{0.2pt}}

% Links- und PDF-Einstellungen
\usepackage[hidelinks]{hyperref}
\hypersetup{
	pdfauthor = {\verfasser},
	pdftitle = {\titel},
	pdfsubject = {\art},
	pdfkeywords = {},
	pdfstartview = {Fit},
	colorlinks = {false},
	breaklinks = {true},
	bookmarksopen = {true}
}

% Verhinderung von Schusterjunge und Hurenkind
\clubpenalty = 10000
\widowpenalty = 10000
\displaywidowpenalty = 10000

% Seitenzäler für große, römische Zahlen
\newcounter{RomanPagenumber}

% Abkürzungen
\newcommand{\dash}{d.\,h.}
\newcommand{\zB}{z.\,B.}
