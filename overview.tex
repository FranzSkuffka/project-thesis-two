\chapter{Übersicht der Funde}
\noindent Die folgenden Testergebnisse wurden mit einer Risikoklasse von 2 oder höher eingestuft. Diese Funde sind zu korrigieren, um die Sicherheit der Anwendung sicher zu stellen.
\newline
\fittingtable{l l X}{
    Schlüssel & Klasse & Ergebnis \\
    \hline
    
AUTHOR-001 & 2 & Passwort für Datenbank wurde nicht verändert \\
AUTHOR-002 & 3 & Nutzer “Admin” kann sich selbst in der Nutzerverwaltung löschen \\
AUTHOR-004 & 2 & php\char`_info.php wurde nicht gelöscht und ist öffentlich einsehbar \\
SESSION-002 & 3 & Session Stealing \\
OTHER-000 & 2 & Protected Ordner ist frei zugänglich \\
}

\pagebreak
\noindent Die folgenden Funde weisen auf Unregelmäßigkeiten hin. Diese Unregelmäßigkeiten sollten behoben werden, beeinflussen allerdings die Sicherheit der Anwendung eher geringfügig.
\newline
\fittingtable{l l X}{
    Schlüssel & Klasse & Ergebnis \\
    \hline
    AUTHEN-000 & 1 & Escaping von Tag-Content und JavaScript \\
    AUTHEN-005 & 1 & Bei der Passwort vergessen Funktion wird bekannt gegeben, ob die Email in der Datenbank exisitiert. \\
    AUTHOR-000 & 1 & “root”-Nutzer von der Datenbank wird genutzt um Operationen auszuführen \\
    AUTHOR-003 & 1 & Nutzerverwaltung des Admin-Nutzers wird nicht vollständig geprüft \\
    AUTHOR-005 & 1 & Seiten die normalerweise nicht als Link verfügbar ist, sind sichtbar \\
    AUTHOR-006 & 1 & Admin Bereich wird durch Redirect verraten (Statuscode: 302 / OK) \\
    SESSION-000 & 1 & Brute Force kann gegen den Cookie angewendet werden (Theoretisch) \\
    INJECTION-000 & 1 & PHP Error-Log öffentlich zugänglich \\
    OTHER-004 & 1 & Debug Modus ist aktiviert \\
    OTHER-005 & 1 & Es wird nicht geprüft ob alle Get-Parameter vorhanden sind \\
}

\pagebreak
\noindent Es wurden einige Tests durchgeführt, die keine Funde ergaben. Diese Ergebnisse zeigen, in welchen Bereichen die Anwendung sicher ist und dem Angreifer keine Möglichkeit bietet, das System zu manipulieren.
\newline
\fittingtable{l l X}{
    Schlüssel & Klasse & Ergebnis \\
    \hline
    AUTHEN-001 & 0 & Kein Brute Force gegen den Login möglich \\
    AUTHEN-002 & 0 & SHA-1 Hash ist bei Passwörtern mit 8 Zeichen, Groß- und Kleinschreibung und min. einer Zahl sicher \\
    AUTHEN-003 & 0 & Email zur Bestätigung nur einmal verwendbar \\
    AUTHEN-004 & 0 & Logg-out nach 20 Minuten ohne Nutzer-Interaktion \\
    AUTHEN-006 & 0 & Kron-Jobs funktionieren nicht wegen falscher Datenbank- Authentifizierung  \\
    SESSION-001 & 0 & Nach dem Login erhält der Nutzer eine neue Sesion-ID \\
    INJECTION-001 & 0 & Keine Eingaben können mit Schadcode ausgestattet werden  \\
    INJECTION-002 & 0 & Eingaben von Nutzerdaten werden immer mit dem Request-Operator “POST” versendet \\
    OTHER-001 & 0 & Captch verhindert das Senden von Span Nachrichten durch die Anwendung \\
    OTHER-002 & 0 & XSS und CSRF-Validierung wurden aktiviert \\
    OTHER-003 & 0 & HTTP Only Atribut ist gesetzt \\
}