\chapter{Grundlagen}
Nach der Definition der Anforderungen werden nun die theoretischen Grundlagen vorgestellt, auf die bei der Entwicklung des Systems genutzt werden. Danach werden die wissenschaftlichen Felder beschrieben, die sich mit den Problemen befassen, die auf dem Weg von der Eingabe eines Textes zur Ausgabe eines Datensatzes liegen. Hierbei gehen wir jedoch in umgekenrter Reihenfolge vor. Das heißt, dass wir zunächst das Feld der \gls{ie} betrachten werden. Diese Grundlage wird später genutzt, um die Vorlage, die ein einzelner Datentyp bereitstellt, zu befüllen. \gls{ie} baut auf das Feld der \gls{ae} auf, die sich mit der Erkennung von Worten innerhalb von Texten befasst. \gls{ae} nutzt wiederum auf das Feld der Klassifikation. Das ziel der Klassifikation ist im Kontext dieser Projektarbeit auch für die Feststellung des Korrekten Datentyps und folglich der Auswahl der Vorlage zuständig. Dabei nutzt die Klassifikation so wie \gls{ie} die möglichkeiten der \gls{ae} um Merkmale eines einzelnen Textes zu erkennen. Wir beschreiben jedoch zunächst die Charakteristika der Problemdomäne \gls{crm} und wie sie sich nutzen lassen.


\section{Domäne CRM}
\gls{crm} ist eine Strategie zur Pflege der Beziehung eines Unternehmens mit dessen Kunden. Sie integriert Personal, Prozesse und Technologien zur Pflege, Erhaltung und von Kundenbeziehungen. Ziel ist die Maximierung der Einnahmen durch Steigerung der Kundenzufriedenheit. Dabei wird der Kunde konzeptionell in den Mittelpunkt aller Geschäftsprozesse gestell. \gls{crm} kann von Unternehmen der Bereiche \gls{b2c} und \gls{b2b} eingesetzt werden.\cite{chen2003understanding} Aus dem Konzept, den Kunden als Dreh- und Angelpunkt der Geschäftsprozesse zu betrachten, lässt sich ein konkretes Modell für die Strukturierung der erfassten Daten ableiten. Jenes Modell, das als Grundlage für diese Projektarbeit dient, definiert, dass jeder Datensatz, der selbst keinen Kunden repräsentiert, dem Datensatz eines Kunden zugewiesen ist. \cite{puckey2001modeling} Wir konzentrieren uns in dieser Projektarbeit auf den \gls{b2b} Bereich, um die Erfassung der Daten zu vereinfachen.
relation?

\section{\Acrfull{ae}}
\gls{ae} bezeichnet die Aufgabe, z.B. Informationseinheiten wie Namen, Organisation und Ortsbezeichnungen oder numerische Ausdrücke wie Zeiten, Datumsangabgen oder Prozentangaben aus Texten zu extrahieren und also solche zu kennzeichnen. \gls{ae} ist primär eine Unteraufgabe der Informationsextraktion\cite{manning2012information}. Um diese benannten Entitäten zu erhalten sind zwei Schritte erforderlich: Zunächst wird der Text vorverarbeitet. Das Ergebnis dieses Schrittes ist im besten Fall ein Sammlung aller nicht mehr sinnvoll zu Teilenden Textbestandteile. Diese Bestandteile, auch Tokens genannt, sind häufig einzelne Worte, also durch Leer- oder Satzzeichen getrennte Zeichenketten, beschränken sich allerdings nicht darauf. Der Name einer Person z.B. kann in einem aber auch mehreren Worten bestehen. Im folgenden werden die Merkmale der einzelnen Tokens bestimmt.\footnote{\cite{nadeau2007survey}}

\section{Extraktion von Information}
Informationsextraktion befasst sich mit dem Problem, relevante Information aus Texten zu extrahieren. Dabei wird die Struktur, meist das Schema für einen Datensatz bzw. Datenbankeintrag vorgegeben, die die extrahierten Daten annehmen sollen. Diese Struktur zeichnet sich als eine Zusammenstellung von typisierten Feldern aus. Im folgenden sind die Schritte aufgeführt, die durchgeführt werden, um diese Felder mit Information zu befüllen.
\begin{enumerate}
	\item Segmentierung
	\item Classification
	\item Association
	\item Deduplication
\end{enumerate}
\cite{mccallum2005information}
Segmentation 
Dayne Freitag description
differenzierung von einzelnen slots
Es wird davon ausgegangen, dass im System Kategorien bzw. Klassen definiert sind, die jeweils ein bestimmtes Schema bzw. eine Vorlage vorgeben. Diese Vorlage bestimmt die Struktur, also die einzelnen Felder einer Instanz dieser Klasse. Eine Instanz stellt hier einen im System zu speichernden Datensatz dar. Als Extraktion wird der Vorgang bezeichnet, bei dem einzelne Entitäten, also die Werte eines Formulars, aus der Eingabe gewonnen werden.


\section{Klassifikation}
D. Michie  D. J. Spiegelhalter  and C. C. Taylor beschreiben Klassifikation im Allgemeinen als Aufgabe, Vorhersagen oder Entscheidungen auf Basis momentan verfügbarer Information zu treffen. In "Machine Learning, Neural and Statistical Classification" beschränken sie diese Aufgabe auf die Einordnung eines Falles, oder im Kontext dieser Projektarbeit auch Objekt genannt, in eine von mehreren vorgegeben Kategorien. Wenn in dieser Projektarbeit von Klassifikation die Rede ist, dann ist auch die zuletzt beschriebene Aufgabe gemeint. Diese Art der Klassifikation, also die Einordnung einer Objekt in eine einzelne Kategorie bzw. die Zuordnung zu einer spezifischen Klasse wird als Multiclass Classification bezeichnet. In gängiger Literatur ist zudem von zwei weiteren Formen der Klassifikation die Rede: Binäre Klassifikation, bei der eine Objekt in eine von genau zwei Kategorien eingeordnet wird und Multilabel Classification, bei der einer Objekt keine bis mehrere Klassen bzw. "Labels" zugewiesen werden.\cite{michie1994machine}
In der zuletzt erwähnten Publikation werden drei verschiedene Methoden zur Klassifikation beschrieben.
Um Entitäten in einzelne Klassen einzuordnen, müssen letztere als ein Satz von Merkmalen definiert werden. Merkmale sind Schlagwörter oder charakteristische Eigenschaften, die dazu konstruiert sind, algorhithmisch weiterverarbeitet zu werden. \cite{nadeau2007survey} Die Ausprägung dieser Merkmale bei den einzelnen Entitäten wird dann genutzt, um möglichst genau die Zugehörigkeit einer Entität zu einer Klasse zu bestimmen.

\section{Maschinelles Lernen}
Die bis hier beschriebenen Formen des maschinellen Lernens sind als überwachtes Lernen zu bezeichnen. Das heißt, dass die Klassen, in die die Objekte eingeordnet werden sollen, bereits vorgegeben sind. Beim unüberwachten Lernen hingegen wird die Verteilung der einzelnen Objekte im Merkmalsraum betrachtet. Anhand dieser Verteilung werden die Objekte dann entsprechend ihrer Gemeinsamkeiten und Unterschiede in Gruppen eingeteilt. Die Einteilung in Gruppen ohne vorgegebene Klassen wird als Clustering bezeichnet.
Machine Learning
Neurale Netzwerke

\cite{nadeau2007survey}
Um zu dem Resultat: 2 Schritte Vorverarbeitung, also Extraktion von Entitäten und 
Bias: input collection
Noise
Classification Algorithm
\section{Textformat}
Präzision
keine Prosa
informell
heuristische Methoden die speziell auf die Problemdomäne zugeschnitten sind.
welche Teile des systems werden modelliert etc.
\section{Maschinelles Lernen}
\section{Merkmalsbildung}
part of preprocessing
Merkmale / Merkmalsraum
\section{Plattform}

