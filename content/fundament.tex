\chapter{Grundlagen}

\section{Domäne CRBM}
Customer Relationship Management stellt den Kunden in den Mittelpunkt aller Geschäftsprozesse. CRM Software kann dazu genutzt werden, geschäftsrelevante Information zu erfassen, auszugeben und zu analysieren. Für die Strukturierung der erfassten Daten kann somit auch ein Modell genutzt werden, dass den Kunden, also den Datensatz eines einzelnen Geschäftskontakts in den Mittelpunkt stellt. Dieses Modell, das als Grundlage für diese Projektarbeit genutzt wird definiert, dass jeder Datensatz, der selbst keinen Kunden darstellt, dem Datensatz eines Kunden zugewiesen ist.

\section{Klassifikation}
D. Michie  D. J. Spiegelhalter  and C. C. Taylor beschreiben Klassifikation im Allgemeinen als Aufgabe, Vorhersagen oder Entscheidungen auf Basis momentan verfügbarer Information zu treffen. In "Machine Learning, Neural and Statistical Classification" beschränken sie diese Aufgabe auf die Einordnung eines Falles, oder im Kontext dieser Projektarbeit auch Entität genannt, in eine von mehreren vorgegeben Kategorien. Wenn in dieser Projektarbeit von Klassifikation die Rede ist, dann ist auch die zuletzt beschriebene Aufgabe gemeint \cite{michie}. Diese Art der Klassifikation, also die Einordnung einer Entität in eine einzelne Kategorie bzw. die Zuordnung zu einer spezifischen Klasse wird als Multiclass Classification bezeichnet. In gängiger Literatur ist zudem von zwei weiteren Formen der Klassifikation die Rede: Binäre Klassifikation, bei der eine Entität in eine von genau zwei Kategorien eingeordnet wird und Multilabel Classification, bei der einer Entität keine bis mehrere Klassen bzw. "Labels" zugewiesen werden. 
In der zuletzt erwähnten Publikation werden drei verschiedene Methoden zur Klassifikation beschrieben.
Die bis hier beschriebenen Formen des maschinellen Lernens sind als überwachtes Lernen zu bezeichnen. Das heißt, dass die Klassen, in die die Entitäten eingeordnet werden sollen, bereits vorgegeben sind. Beim unüberwachten Lernen hingegen wird die Verteilung der einzelnen Entitäten im Merkmalsraum betrachtet. Diese werden dann entsprechend ihrer Gemeinsamkeiten und Unterschiede in Gruppen eingeteilt. Die einteilung in Gruppen ohne vorgegebene Klassen wird als Clustering bezeichnet.
Machine Learning
Neurale Netzwerke
Um Entitäten in einzelne Klassen einzuordnen, müssen letztere als ein Satz von Merkmalen definiert werden. Die Ausprägung dieser Merkmale bei den einzelnen Entitäten wird dann genutzt, um möglichst genau die Zugehörigkeit einer Entität zu einer Klasse zu bestimmen.
Nach der Erläuterung des Aufgabenbereichs der Klassifikation wird nun das feld der NER wird betrachtet, welches auch die Aufgabe der Klassifikation beinhaltet.
Im folgenden betrachten 


\section{Named Entity Recognition}
 bezeichnet die Aufgabe, z.B. Informationseinheiten wie Namen, Organisation und Ortsbezeichnungen oder numerische Ausdrücke wie Zeiten, Datumsangabgen oder Prozentangaben aus Texten zu extrahieren. Um diese benannten Entitäten zu erhalten sind zwei Schritte erforderlich: Zunächst wird der Text vorverarbeitet. Das Ergebnis dieses Schrittes ist im besten Fall ein Sammlung aller nicht mehr sinnvoll zu Teilenden Textbestandteile. Diese Bestandteile, also Entitäten, sind häufig einzelne Worte, also durch Leerzeichen getrennte Zeichenketten, beschränken sich allerdings nicht darauf. Der Name einer Person z.B. kann in einem aber auch mehreren Worten bestehen. Nach dieser 
Um zu dem Resultat: 2 Schritte Vorverarbeitung, also Extraktion von Entitäten und 
Bias: input collection
Noise
Classification Algorithm
\section{Extraktion von Information}
DAYNE FREITAG beschreibt Informationsextraktion als "Slot-Filling-Problem" \cite{Freitag}.
Dayne Freitag description
differenzierung von einzelnen slots
Es wird davon ausgegangen, dass im System Kategorien bzw. Klassen definiert sind, die jeweils ein bestimmtes Schema bzw. eine Vorlage vorgeben. Diese Vorlage bestimmt die Struktur, also die einzelnen Felder einer Instanz dieser Klasse. Eine Instanz stellt hier einen im System zu speichernden Datensatz dar. Als Extraktion wird der Vorgang bezeichnet, bei dem einzelne Entitäten, also die Werte eines Formulars, aus der Eingabe gewonnen werden.
\section{Unstrukturierte Texte}

heuristische Methoden die speziell auf die Problemdomäne zugeschnitten sind.
welche Teile des systems werden modelliert etc.
