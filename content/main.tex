\chapter{Einleitung}
\section{Problemstellung}
Eingabeformulare als Barriere
Konzept extraktion und separate speicherung als Datensatz
\section{Motivation}
non-tech affinity \cite{hemken}
bei großen system fällt Überblick schwer.
human indexing is expensive, cost reduction because expert-labour \cite{shneiderman}
\chapter{}
\section{Funktionale Anforderungen}
Es wird davon ausgegangen, dass im System Kategorien bzw. Klassen definiert sind, die jeweils ein bestimmtes Schema bzw. eine Vorlage vorgeben. Diese Vorlage bestimmt die Struktur, also die einzelnen Felder einer Instanz dieser Klasse. Eine Instanz stellt hier einen im System zu speichernden Datensatz dar. Als Extraktion wird der Vorgang bezeichnet, bei dem einzelne Entitäten, also die Werte eines Formulars, aus der Eingabe gewonnen werden.
\begin{enumerate}
\item Das System soll die Eingabe in Textform entgegen nehmen.
\item Das System soll jeweils eine Eingabe in eine Vorgegebene Kategorie einordnen.
\item Danach soll das System die durch die Kategorie bestimmte Vorlage mit aus der Eingabe extrahierten Informationen befüllen.
\end{enumerate}


\chapter{Grundlagen}
Modelle
Klassen aus CRM Bereich, problemwiederholung: einordnung. Das heißt:
domäne crm, ausgewählte kategorien
Technologien
Javascript

Klassifikation
wie haben andere die probleme gelöst
Erklärung Klassifikation: Zwei typen: supervised vs unsupervised

Extraktion

\chapter{Theoretische Herangehensweise}
Nach der Definition der funktionalen Anforderungen an das System wird nun die theoretische Herangehensweise beschrieben. Dabei wird die Funktion des Systems in nacheinander geschaltete Mechanismen geteilt.
\section{Routing}
Klassen sind vorgegeben, multiclass classification weil CITE WIKIPEDIA
Ziel: Datensatz
\section{Merkmale}
Bei der Entwicklung der Vorgehensweise zur Klassifikation der eingegebenen Texte wird davon ausgegangen, dass keine exogene Information zur Verfügung steht. Das heißt, dass der Klassifikationsmechanismus nicht auf Metadaten wie z.B. Autor oder Uhrzeit zugreifen kann. Somit steht nur endogene Information, also der Inhalt der eingegebenen Texte zur Verfügung. Die Prüfung des Vorhandenseins von Merkmalen beschränkt sich also auf Bereich des Textkörpers.\cite[sebastiano]

characterization - finding features of certain category
Tabellenspalte > Absoluter Wert, Querverweis, Spaltenname
Merkmale definiert durch Datenbankschema, in das abgespeichert wird. Die Einzelnen Datenbankschemata lassen sich anhand der Felder, die sie definieren, unterscheiden. Somit ist das Vorhandensein eines solchen Feldes. Ein solches Schema definiert Felder, die absolute Werte oder Querverweise enthalten können. 

characteration through template?
Absoluter Wert = Entity / Pattern recognition
Querverweis = String matching
Spaltenname = keyword search
Klasse > Merkmale
Merkmalsraum?


\section{Klassifikation}
Merkmale > Classifier
Verarbeitung: Eingabe > Merkmale > Konfidenzintervall
art des merkmals 
Machine Learning


\section{Extraktion von Information}
Merkmale die vorhanden sind können weiter verwendet werden.
Verstärkte suche
lernprozess - learning when something was missed if user enters it and it is also found inside the text

\chapter{Konzeption}
technologische Grundlagen beachten
alchemy api
learning classifiers
regular expressions
scope der arbeit beachten
größe des Trainingdatensatzes beachten
analogie zu theoretischer herangehensweise
problem des trainingsdatensatz: bereits exportierte, verfremdete daten
\chapter{Implementierung}
\section{Klassifikation der Eingabe}
\section{Extraktion von Information}

\chapter{Schluss}
learning by doing encouraged. people get introduced to the problem domain throught the suggestions served by the system.


\chapter{optional}
überlegungen zur benutzeroberfläche
teilautomatische "interaktive" klassifikation und extraction, weil: trainingsdaten rar? und vertrauen auf den classifier gering.
\subsection{Datenverarbeitung}
automatisierte generierung von metadaten, metadaten werden hier das neue zentrum der interessen
exogenous information - phone note for example on ingoing phone call
echte anforderung: email-footer
extensible: insert own types with certain features?
fokus auf texte die einer klasse entsprechen
a priori?
Simon Mayer becomes Simon Mayer (recognised kind: account)
\section{Theoretische Herangehensweise}
Texte fast nie grammatikalisch korrekt
endogene Information
problem of notes - bad spelling
\subsection{Klassifikation der Eingabe}
Im folgenden wird die theoretische Herangehensweise an die Klassifikation der eingegebenen Daten beleuchtet.
Categorization - using chrome? for email footer for example
Occam's Razor?


\chapter{Schluss}


section{To Do}
Jones fragen
