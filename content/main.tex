\chapter{Einleitung}
\section{Problemstellung}
Eingabeformulare als Barriere

% Motivation
human indexing is expensive, cost reduction because expert-labour \cite{shneiderman}
non-tech affinity 
bei großen system fällt Überblick schwer.




\chapter{Grundlagen}
automatisierte generierung von metadaten, metadaten werden hier das neue zentrum der interessen
Simon Mayer becomes Simon Mayer (recognised kind: account)
Texte fast nie grammatikalisch korrekt
problem of notes - bad spelling
\section{Domäne CRM}
Diese Projektarbeit konzentriert sich auf Texte und Kategorien der Domäne CRM. Das bedeutet nicht, dass das entwickelte System nur in der Lage ist, die funktionalen Anforderungen nur in jener Domäne zu erfüllen. Im Gegenteil: das System sollte so flexibel sein, dass es auch für Informationssysteme genutzt werden kann, deren Entitäten sowohl an sich als auch untereinander ähnlich strukturiert sind.
\section{Klassifikation}
D. Michie  D. J. Spiegelhalter  and C. C. Taylor beschreiben Klassifikation im Allgemeinen als Aufgabe, Vorhersagen oder Entscheidungen auf Basis momentan verfügbarer Information zu treffen. In "Machine Learning, Neural and Statistical Classification" beschränken sie diese Aufgabe auf die Einordnung eines Falles, oder im Kontext dieser Projektarbeit auch Entität genannt, in eine von mehreren vorgegeben Kategorien. Wenn in dieser Projektarbeit von Klassifikation die Rede ist, dann ist auch die zuletzt beschriebene Aufgabe gemeint.\cite[michie] Diese Art der Klassifikation, also die Einordnung einer Entität in eine einzelne Kategorie bzw. die zuordnung zu einer spezifischen Klasse wird als Multiclass Classification bezeichnet. In gängiger Literatur ist zudem von zwei weiteren Formen der Klassifikation die Rede: Binäre Klassifikation, bei der eine Entität in eine von genau zwei Kategorien eingeordnet wird und Multilabel Classification, bei der einer Entität keine bis mehrere Klassen bzw. "Labels" zugewiesen werden.


Bei der Entwicklung der Vorgehensweise zur Klassifikation der eingegebenen Texte wird davon ausgegangen, dass keine exogene Information zur Verfügung steht. Das heißt, dass der Klassifikationsmechanismus nicht auf Metadaten wie z.B. Autor oder Uhrzeit zugreifen kann. Somit steht nur endogene Information, also der Inhalt der eingegebenen Texte zur Verfügung. Die Prüfung des Vorhandenseins von Merkmalen beschränkt sich also auf Bereich des Textkörpers.\cite[sebastiano]


wie haben andere die probleme gelöst
beschreiben drei verschiedene Ansätze zur Klassifikation 
Statistische Herangehensweise. Menschlicher eingriff nötig
Machine Learning
Neurale Netzwerke
Erklärung Klassifikation: Zwei typen: supervised vs unsupervised
a priori?
Occam's Razor?
No feature Selection and extraction
\section{Extraktion}

Es wird davon ausgegangen, dass im System Kategorien bzw. Klassen definiert sind, die jeweils ein bestimmtes Schema bzw. eine Vorlage vorgeben. Diese Vorlage bestimmt die Struktur, also die einzelnen Felder einer Instanz dieser Klasse. Eine Instanz stellt hier einen im System zu speichernden Datensatz dar. Als Extraktion wird der Vorgang bezeichnet, bei dem einzelne Entitäten, also die Werte eines Formulars, aus der Eingabe gewonnen werden.


\chapter{Funktionale Anforderungen}
\begin{enumerate}
\item Das System soll die Eingabe in Textform entgegen nehmen.
\item Das System soll jeweils eine Eingabe in eine Vorgegebene Kategorie einordnen.
\item Danach soll das System die durch die Kategorie bestimmte Vorlage mit aus der Eingabe extrahierten Informationen befüllen.
\end{enumerate}
\chapter{Theoretische Herangehensweise}
Nach der Definition der funktionalen Anforderungen an das System wird nun die theoretische Herangehensweise beschrieben. Dabei wird die Funktion des Systems in nacheinander geschaltete Mechanismen geteilt.

\section{Klassifikation}
Klassen sind vorgegeben, multiclass classification weil CITE WIKIPEDIA
Ziel: Datensatz
% Merkmale

characterization - finding features of certain category
Tabellenspalte > Absoluter Wert, Querverweis, Spaltenname
Merkmale definiert durch Datenbankschema, in das abgespeichert wird. Die Einzelnen Datenbankschemata lassen sich anhand der Felder, die sie definieren, unterscheiden. Somit ist das Vorhandensein eines solchen Feldes. Ein solches Schema definiert Felder, die absolute Werte oder Querverweise enthalten können. 

characteration through template?
Absoluter Wert = Entity / Pattern recognition
Querverweis = String matching
Spaltenname = keyword search
Klasse > Merkmale
Merkmalsraum?


Merkmale > Classifier
Verarbeitung: Eingabe > Merkmale > Konfidenzintervall
art des merkmals 
Machine Learning

\section{Extraktion von Information}
Merkmale die vorhanden sind können weiter verwendet werden.
Verstärkte suche
using chrome? for email footer for example
lernprozess - learning when something was missed if user enters it and it is also found inside the text





\chapter{Konzeption}
Konzept extraktion und separate speicherung als Datensatz
technologische Grundlagen beachten
alchemy api
learning classifiers
regular expressions
scope der arbeit beachten
größe des Trainingdatensatzes beachten
analogie zu theoretischer herangehensweise
problem des trainingsdatensatz: bereits exportierte, verfremdete daten
\chapter{Implementierung}
\section{Klassifikation der Eingabe}
\section{Extraktion von Information}

\chapter{Schluss}
learning by doing encouraged. people get introduced to the problem domain throught the suggestions served by the system.


\chapter{optional}
% UI
teilautomatische "interaktive" klassifikation und extraction, weil: trainingsdaten rar? und vertrauen auf den classifier gering.
echte anforderung: email-footer




section{To Do}
Jones fragen
transfer literature to bibfile
check formal requirements
