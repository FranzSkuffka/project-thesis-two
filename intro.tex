\chapter{Kurzbeschreibung der Webanwendung}
Bei der WebApp “Bolzplatz” handelt es sich um ein Fußball-Tippspiel. Nach erfolgreicher Registrierung und Anmeldung hat man die Möglichkeit, sich an einem Tippspiel zu beteiligen. Außerdem besteht die Möglichkeit Tippgruppen, in denen man mit anderen Spielern wetten kann, zu erstellen. Ist der Tipp richtig, so bekommt man Punkte. Es gibt eine Highscore-Tabelle mithilfe welcher man sich mit anderen Spielern vergleichen kann. Auch die verschiedenen Tippgruppen können miteinander verglichen werden.
\chapter{Management-Summary}
\noindent Bei der Sicherheitanalyse der Anwendung sind Mängel festgestellt worden.

Bei der Anwendung wird durchweg auf HTTP gesetzt (AUTHEN-000). Dadurch ist eine Man-in-the-middle-Attack möglich. Das System sollte von HTTP auf HTTPS umgestellt werden. Der Admin-Nutzer kann sich selbst in der Anwendung löschen (AUTHOR-002). Anschließend lässt sich ein neuer Admin-Nutzer registrieren, welcher dann alle Rechte besitzt. Das sollte in keinem Fall möglich sein.
Das Passwort der Datenbank wurde nicht verändert (AUTHOR-001). Dadurch erhält man mit den Default Credentials (root/vagrant) unbeschränkten Datenbankzugriff. Dort sind alle Passwörter in verschlüsselter Form vorhanden. Unter index.php unterhalb des Bolzplatz-Ordners ist die php\char`_info.php erreichbar (AUTHOR-004). Diese zeigt die vollständigen Server-Konfigurationen, welche mögliche Sicherheitslücken für den Angreifer erkennbar macht.
Sobald sich ein Nutzer eingeloggt bekommt dieser eine neue Session-ID. Diese Session-ID kann allerdings mit den Chrome Developer Tools ausgelesen und versendet werden. Dadurch ist es möglich, sich ohne Passwort und Nutzernamen in einen fremden Account einzuloggen (SESSION-002).

Die auf den drei Grundpfeilern Vertraulichkeit, Integrität und Verfügbarkeit ruhende Datensicherheit der Webanwendung ist im aktuellen Stand gefährdet. Es wird empfohlen, mindestens die im Dokument mit Risikoklasse 2 und höher eingestuften Mängel zu beseitigen.

\chapter{Risikoklassen und Schlüssellegende}
Um die Testergebnisse einzuordnen, werden Klassen und Schlüssel definiert. Die Klassen stellen dabei die Größe des Sicherheitsrisikos dar. Je höher die Zahl der Risikoklasse um so höher ist die Auswirkung auf die Sicherheit der Anwendung. Die Schlüssel beschreiben verschiedene Begriffe, welche dazu dienen sollen, die Tests einer bestimmten Kategorie zuzuordnen.
\section{Risikoklassen}
Alle Testergebnisse werden in die folgenden 4 Risikoklassen eingestuft:
\newline
\addvbuffer[\tabularspacing]{

    \begin{tabular}{l|l l}
    0 & Information & Kein Risiko, rein informativ \\
    1 & Hinweis & Hinweis auf Unregelmäßigkeit, die behoben werden sollte \\
    2 & Problem & Zu korrigierender Mangel \\
    3 & Kritisch & Kritischer Mangel \\
    \end{tabular}
}


\section{Schlüssellegende}
Für die Kategorisierung wurden folgende Schlüssel zugrunde gelegt:
\newline
\addvbuffer[\tabularspacing]{
    \begin{tabular}{l l}
    AUTHEN & Authentifizierungsschwachstellen \\
    AUTHOR & Autorisierungsschwachstellen \\
    SESSION & Session-Schwachstellen \\
    INJECTION & Injektionsschwachstellen \\
    OTHER & Andere Schwachstellen \\
    \end{tabular}
}
\newline
Je nachdem, welche Kategorie der Test anspricht, wird der Test benannt. Die Namensgebung der Tests erfolgt nach einem festgelegten Muster:
{Schlüsselname}-{Nummer des Tests} 
Beispiel: SESSION-001
Dadurch werden die Tests eindeutig von einander abgegrenzt.
