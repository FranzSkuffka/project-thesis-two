% arara: xelatex
%: {options: -output-directory=./output}
	% Hinweis: Optionen der Dokumentenklasse werden an alle folgenden \usepackage{package} Befehle weitergegeben
\documentclass[
	fontsize=12pt,
	paper=a4,
	parskip=half,
	twoside=false,
	numbers=noenddot,	% Kein Punkt am Ende einer Überschrift
	%draft=true,			% Deckt Schwächen auf: overfull und full boxes werden markiert; Bilder werden nicht geladen
	bibliography=totoc,	% Literaturverzeichnis ins Inhaltsverzeichnis aufnehmen
	listof=totoc,		% Tabellen- und Abbildungsverzeichnis ins Inhaltsverzeichnis aufnehmen
	titlepage=true,		% Separate Titelseite; Gestaltung mit Hilfe der Titlepage-Umgebung
	headsepline=true,	% Kopflinie aktivieren
	footsepline=true,	% Fußlinie aktivieren
	abstracton			% Abstract aktivieren
]{scrreprt}

% Zeichenkodierung Ausgabe ist T1-Kodierung: Wichtig für die Ausgabe von Umlauten
\usepackage[T1]{fontenc}

% Schrift festlegen
\usepackage{fontspec}

\setmainfont[
BoldFont=Arial Bold.ttf,
ItalicFont=Arial Italic.ttf,
BoldItalicFont=Arial Bold Italic.ttf
]{Arial.ttf}

\usepackage{titlesec}
\renewcommand{\sectfont}{\rmfamily}
% Sprachauswahl für Lokalisierungen und Silbentrennung
\usepackage[german]{babel}

% Zitate: Anführungszeichen automatisch anhand der Sprache wählen
\usepackage[babel=true]{csquotes}

% Source-Code-Listings
\usepackage{listings}
\lstset{
    language=SQL,
    basicstyle=\ttfamily
} 

% BibTeX-Symbol
\usepackage{texnames}

%Farbpaket laden
\usepackage{xcolor}

% Symbole, z.B. Haken
\usepackage{pifont}

% Zeilen in Tabellen zusammenfassen
\usepackage{multirow}

% Silbentrennung kann bei bestimmten Wörten mit Hilfe von diesem Paket deaktiviert werden 
\usepackage{hyphenat}

% Abkürzungsverzeichnis
\usepackage[printonlyused, withpage]{acronym}
% Abstand mit Punkten füllen

% Tiefe des Inhaltsverzeichnisses
\setcounter{tocdepth}{2}

% Punkte im Inhaltsverzeichnis
\usepackage{tocstyle}
\usetocstyle{allwithdot}

% Zum Einbinden von PDF-Dateien.
\usepackage{pdfpages}

% Paket zum Anpassen von Kopf- und Fußzeilen
\usepackage[plainfootsepline, plainheadsepline, headsepline, footsepline, automark]{scrpage2}
\setlength{\headheight}{1.1\baselineskip}
% Liniendicke
\setheadsepline{0.1pt}
\setfootsepline{0.1pt}

% Kopf- und Fusszeile löschen
\clearscrheadfoot
% Kopf- und Fusszeile aktivieren
\pagestyle{scrheadings}

% Kopf links
\ihead[\titel]{\titel}

% Fuss links
\ifoot[\verfasser]{\verfasser}
% Fuss rechts
\ofoot[\pagemark]{\pagemark}

% Grafiken einbinden
\usepackage{graphicx}
% Pfad zu den Grafiken
\graphicspath{{imagery/}}

% Seitenränder setzen
\usepackage[left=3.5cm, right=2.5cm, top=2.5cm, bottom=3cm]{geometry}
\renewcommand*\chapterheadstartvskip{\vspace*{0cm}}

% Zeilenabstand auf 1.5 setzen
\usepackage{setspace}
\onehalfspacing

% Literaturverzeichnis
%\usepackage[backend=bibtex,style=authoryear]{biblatex}
\usepackage[backend=bibtex,style=alphabetic]{biblatex}
\bibliography{./literature/Literature.bib}
\renewcommand{\bibname}{Literaturverzeichnis}
%Referenz zu URLs
\usepackage{url}

% Glossar
\usepackage[acronym,toc,nonumberlist]{glossaries}
\makeglossaries

% Titel als Referenzierung verwenden
\usepackage{titleref}

% Währungen
\usepackage{textcomp}

% Fussnoten fortlaufend nummerieren.
\usepackage{chngcntr}
\counterwithout{footnote}{chapter}

% Persönliche Daten
\newcommand{\titel}{Implementierung eines Systems zur Klassifikation von und Extraktion von Information aus Texten der Domäne CRM}
\newcommand{\art}{Projektarbeit des zweiten Studienjahres}
\newcommand{\studienbereich}{Wirtschaft}
\newcommand{\studiengang}{Onlinemedien}
\newcommand{\verfasser}{Jan Wirth}
\newcommand{\kurs}{ON13}
\newcommand{\ausbildungsbetrieb}{visual4 GmbH}
\newcommand{\betreuer}{Prof. Dr. Arnulf Mester}
\newcommand{\abgabedatum}{}
\newcommand{\unterschrift}{\rule{5cm}{0.2pt}}

% Links- und PDF-Einstellungen
\usepackage[hidelinks]{hyperref}
\hypersetup{
	pdfauthor = {\verfasser},
	pdftitle = {\titel},
	pdfsubject = {\art},
	pdfkeywords = {},
	pdfstartview = {Fit},
	colorlinks = {false},
	breaklinks = {true},
	bookmarksopen = {true}
}

% Verhinderung von Schusterjunge und Hurenkind
\clubpenalty = 10000
\widowpenalty = 10000
\displaywidowpenalty = 10000

% Seitenzäler für große, römische Zahlen
\newcounter{RomanPagenumber}

% Abkürzungen
\newcommand{\dash}{d.\,h.}
\newcommand{\zB}{z.\,B.}

% Zitate in neue Zeile rücken
\usepackage{breakcites}

% Glossary capitalization
\usepackage{mfirstuc}
% \renewcommand{\glsnamefont}[2][]{\capitalisewords{#1}\xspace#2}


\begin{document}
	% Title page
	\thispagestyle{plain}

\begin{titlepage}
	\enlargethispage{4.0cm}

	\begin{center}		
		\raisebox{-.5\height}{\includegraphics[width=3cm]{imagery/visual4}}
		\hfill
		\raisebox{-.5\height}{\includegraphics[width=6cm]{imagery/dhbw_mosbach}}
		\\
		\vspace{3cm}
		\large{\textbf{Fachbereich: \studienbereich}}\\
		\vspace{0.5cm}
		\large{\textbf{Studiengang: \studiengang}}\\
		\vspace{1.0cm}
		\Large{\textsc{\textbf{\titel}}}\\
		\vspace{1.0cm}
		\large{\textbf{\art}}\\
		\vspace{1cm}
		\vspace{1cm}
		\vspace{3cm}
		\begin{tabular}{rl}
			Autor:					& \verfasser\\
			Kurs: 					& \kurs\\ 
			Ausbildungsbetrieb:		& \ausbildungsbetrieb\\ 
            Wiss. Ansprechpartner:  & \betreuer\\
			Abgabedatum:		& \abgabedatum\\
			\vspace{0.5cm}\\
			Unterschrift:				& \unterschrift\\
		\end{tabular} 
	\end{center}
\end{titlepage}

	
	% Ab hier große, römische Seitenzahlen
	\pagenumbering{Roman}
	
	%falls nicht der Titel der Arbeit oben stehen soll, bitte einkommentieren
	%\ihead[\headmark]{\headmark}
	
	% Sperrvermerk
	%\include{content/Sperrvermerk}
	
	% Abstract
	%\include{content/Abstract}
	
	% Eidesstattliche Erklärung
	\include{content/Erklaerung}
	
	% Contentssverzeichnis
	\tableofcontents
	\pagebreak
	
	% Abbildungsverzeichnis
	%\listoffigures
	%\pagebreak
    
    % Code Listing Verzeichnis
    %\lstlistoflistings
    %\pagebreak
	
	% Tabellenverzeichnis
	%\listoftables
	%\pagebreak
	
	% Abkürzungsverzeichnis
	%\addchap{List of Abbreviations}
	%\input{content/Abbreviations}
	%\pagebreak
	
	% Abschnitt beenden
	\clearpage
	
	% Seitenzähler erhält den Wert der aktuellen groß römisch nummerierten Seite; Zwischenspeichern für später
	\setcounter{RomanPagenumber}{\value{page}}
	
	% Ab hier arabische Seitenzahlen
	\pagenumbering{arabic}
		
	% Kopf links auf aktuelles Kapitel ändern
	\ihead[\headmark]{\headmark}

	
	\chapter{Einleitung}
Automatisieren ist immer günstiger
\section{Zielgruppe}
UX Designer, Softwareentwickler
Grundwissen; CRM, Softwareentwicklung, Datenbanken, 
\section{Problemstellung}
Wachsende Komplexität
Einfachere Interfaces
Eingabeformulare als Barriere
CRM-Software möglicherweise Hinterher
Efforts von Salesforce

\section{Motivation}
human indexing is expensive, cost reduction because expert-labour \cite{shneiderman}
non-tech affinity 
bei großen system fällt Überblick schwer.

	\chapter{Funktionale Anforderungen}
Als kritischer Faktor der Kostenoptimierung im Betrieb von \gls{crm} Software ist die Effizienz ihrer Nutzer. Dies ist begründet durch die Lernkurve bzw. den Schulungsbedarf den die effektive Nutzung von \gls{crm} Software voraussetzt. Durch Lohnunterschiede  begründet ist mit dem Einsatz von im System geschulten Mitarbeitern ein im Vergleich zu ungeschulten Mitarbeitern  größerer finanzieller Aufwand in derselben Zeit verbunden. Die automatisierung von Arbeitsschritten mit möglichst geringer Fehlerwahrscheinlichkeit ist eine Möglichkeit, diese geschulten Mitarbeiter effizienter arbeiten zu lassen. Bei der Erfassung neuer Datensätze können dem Nutzer mehrere Schritte abgenommen werden.  Hieraus ergeben sich funktionale Anforderungen an das System in Form von linear aufeinander folgenden Teilschritten.
\begin{enumerate}
    \item Das System soll die Eingabe in Textform entgegen nehmen.
    \item Das System soll Konfidenzintervalle für die Zugehörigkeit der Eingabe zu den einzelnen Kategorien ausgeben.
    \item Das System soll die Eingabe der Wahl der zutreffenden Kategorie durch den Nutzer entgegen nehmen.
    \item Das System soll die durch die Kategorie bestimmte Vorlage mit aus der Eingabe extrahierten Informationen befüllen.
\end{enumerate}

Weitere Anforderungen ergeben sich aus dem Ziel, das System eine tatsächlich \gls{crm} Software integrieren zu können. Auch ist der Einsatz zunächst auf den deutschsprachigen Raum und die Domäne CRM fokussiert. 
Da das System in tatsächliche \gls{crm} Software integriert können soll, werden seitens der Softwareentwickler weitere Anforderungen definiert.
\begin{itemize}
    \item Das System soll Texte in deutsch verarbeiten können.
    \item Das System soll in der Programmiersprache \gls{js} entwickelt werden.
    \item Das System soll Schnittstellen für jeden Prozesschritt bieten
\end{itemize}
Nach der Definition der funktionalen Anforderungen werden nun die Grundlagen beschrieben, auf denen das System aufgebaut wird.

	\chapter{Grundlagen}
Nach der Definition der Anforderungen werden nun die theoretischen Grundlagen vorgestellt, auf die bei der Entwicklung des Systems genutzt werden. Danach werden die wissenschaftlichen Felder beschrieben, die sich mit den Problemen befassen, die auf dem Weg von der Eingabe eines Textes zur Ausgabe eines Datensatzes liegen. Hierbei gehen wir jedoch in umgekenrter Reihenfolge vor. Das heißt, dass wir zunächst das Feld der \gls{ie} betrachten werden. Diese Grundlage wird später genutzt, um die Vorlage, die ein einzelner Datentyp bereitstellt, zu befüllen. \gls{ie} baut auf das Feld der \gls{ae} auf, die sich mit der Erkennung von Worten innerhalb von Texten befasst. \gls{ae} nutzt wiederum auf das Feld der Klassifikation. Das ziel der Klassifikation ist im Kontext dieser Projektarbeit auch für die Feststellung des Korrekten Datentyps und folglich der Auswahl der Vorlage zuständig. Dabei nutzt die Klassifikation so wie \gls{ie} die möglichkeiten der \gls{ae} um Merkmale eines einzelnen Textes zu erkennen. Wir beschreiben jedoch zunächst die Charakteristika der Problemdomäne \gls{crm} und wie sie sich nutzen lassen.


\section{Domäne CRM}
\gls{crm} ist eine Strategie zur Pflege der Beziehung eines Unternehmens mit dessen Kunden. Sie integriert Personal, Prozesse und Technologien zur Pflege, Erhaltung und von Kundenbeziehungen. Ziel ist die Maximierung der Einnahmen durch Steigerung der Kundenzufriedenheit. Dabei wird der Kunde konzeptionell in den Mittelpunkt aller Geschäftsprozesse gestell. \gls{crm} kann von Unternehmen der Bereiche \gls{b2c} und \gls{b2b} eingesetzt werden.\cite{chen2003understanding} Aus dem Konzept, den Kunden als Dreh- und Angelpunkt der Geschäftsprozesse zu betrachten, lässt sich ein konkretes Modell für die Strukturierung der erfassten Daten ableiten. Jenes Modell, das als Grundlage für diese Projektarbeit dient, definiert, dass jeder Datensatz, der selbst keinen Kunden repräsentiert, dem Datensatz eines Kunden zugewiesen ist. \cite{puckey2001modeling} Wir konzentrieren uns in dieser Projektarbeit auf den \gls{b2b} Bereich, um die Erfassung der Daten zu vereinfachen.
relation?

\section{\Acrfull{ae}}
\gls{ae} bezeichnet die Aufgabe, z.B. Informationseinheiten wie Namen, Organisation und Ortsbezeichnungen oder numerische Ausdrücke wie Zeiten, Datumsangabgen oder Prozentangaben aus Texten zu extrahieren und also solche zu kennzeichnen. \gls{ae} ist primär eine Unteraufgabe der Informationsextraktion\cite{manning2012information}. Um diese benannten Entitäten zu erhalten sind zwei Schritte erforderlich: Zunächst wird der Text vorverarbeitet. Das Ergebnis dieses Schrittes ist im besten Fall ein Sammlung aller nicht mehr sinnvoll zu Teilenden Textbestandteile. Diese Bestandteile, auch Tokens genannt, sind häufig einzelne Worte, also durch Leer- oder Satzzeichen getrennte Zeichenketten, beschränken sich allerdings nicht darauf. Der Name einer Person z.B. kann in einem aber auch mehreren Worten bestehen. Im folgenden werden die Merkmale der einzelnen Tokens bestimmt.\footnote{\cite{nadeau2007survey}}

\section{Extraktion von Information}
Informationsextraktion befasst sich mit dem Problem, relevante Information aus Texten zu extrahieren. Dabei wird die Struktur, meist das Schema für einen Datensatz bzw. Datenbankeintrag vorgegeben, die die extrahierten Daten annehmen sollen. Diese Struktur zeichnet sich als eine Zusammenstellung von typisierten Feldern aus. Im folgenden sind die Schritte aufgeführt, die durchgeführt werden, um diese Felder mit Information zu befüllen.
\begin{enumerate}
	\item Segmentierung
	\item Classification
	\item Association
	\item Deduplication
\end{enumerate}
\cite{mccallum2005information}
Segmentation 
Dayne Freitag description
differenzierung von einzelnen slots
Es wird davon ausgegangen, dass im System Kategorien bzw. Klassen definiert sind, die jeweils ein bestimmtes Schema bzw. eine Vorlage vorgeben. Diese Vorlage bestimmt die Struktur, also die einzelnen Felder einer Instanz dieser Klasse. Eine Instanz stellt hier einen im System zu speichernden Datensatz dar. Als Extraktion wird der Vorgang bezeichnet, bei dem einzelne Entitäten, also die Werte eines Formulars, aus der Eingabe gewonnen werden.


\section{Klassifikation}
D. Michie  D. J. Spiegelhalter  and C. C. Taylor beschreiben Klassifikation im Allgemeinen als Aufgabe, Vorhersagen oder Entscheidungen auf Basis momentan verfügbarer Information zu treffen. In "Machine Learning, Neural and Statistical Classification" beschränken sie diese Aufgabe auf die Einordnung eines Falles, oder im Kontext dieser Projektarbeit auch Objekt genannt, in eine von mehreren vorgegeben Kategorien. Wenn in dieser Projektarbeit von Klassifikation die Rede ist, dann ist auch die zuletzt beschriebene Aufgabe gemeint. Diese Art der Klassifikation, also die Einordnung einer Objekt in eine einzelne Kategorie bzw. die Zuordnung zu einer spezifischen Klasse wird als Multiclass Classification bezeichnet. In gängiger Literatur ist zudem von zwei weiteren Formen der Klassifikation die Rede: Binäre Klassifikation, bei der eine Objekt in eine von genau zwei Kategorien eingeordnet wird und Multilabel Classification, bei der einer Objekt keine bis mehrere Klassen bzw. "Labels" zugewiesen werden.\cite{michie1994machine}
In der zuletzt erwähnten Publikation werden drei verschiedene Methoden zur Klassifikation beschrieben.
Um Entitäten in einzelne Klassen einzuordnen, müssen letztere als ein Satz von Merkmalen definiert werden. Merkmale sind Schlagwörter oder charakteristische Eigenschaften, die dazu konstruiert sind, algorhithmisch weiterverarbeitet zu werden. \cite{nadeau2007survey} Die Ausprägung dieser Merkmale bei den einzelnen Entitäten wird dann genutzt, um möglichst genau die Zugehörigkeit einer Entität zu einer Klasse zu bestimmen.

\section{Maschinelles Lernen}
Die bis hier beschriebenen Formen des maschinellen Lernens sind als überwachtes Lernen zu bezeichnen. Das heißt, dass die Klassen, in die die Objekte eingeordnet werden sollen, bereits vorgegeben sind. Beim unüberwachten Lernen hingegen wird die Verteilung der einzelnen Objekte im Merkmalsraum betrachtet. Anhand dieser Verteilung werden die Objekte dann entsprechend ihrer Gemeinsamkeiten und Unterschiede in Gruppen eingeteilt. Die Einteilung in Gruppen ohne vorgegebene Klassen wird als Clustering bezeichnet.
Machine Learning
Neurale Netzwerke

\cite{nadeau2007survey}
Um zu dem Resultat: 2 Schritte Vorverarbeitung, also Extraktion von Entitäten und 
Bias: input collection
Noise
Classification Algorithm
\section{Textformat}
Präzision
keine Prosa
informell
heuristische Methoden die speziell auf die Problemdomäne zugeschnitten sind.
welche Teile des systems werden modelliert etc.
\section{Maschinelles Lernen}
\section{Merkmalsbildung}
part of preprocessing
Merkmale / Merkmalsraum
\section{Plattform}


	\chapter{Theoretische Herangehensweise}
Nach der Definition der funktionalen Anforderungen an das System wird nun die theoretische Herangehensweise beschrieben. Dabei wird die Funktion des Systems analog zu den funktionalen Anforderungen in nacheinander geschaltete Mechanismen geteilt.

\section{Klassifikation}
Durch das Ziel, mit der Eingabe einen einem bestimmten Schema entsprechenden Datensatz aufzubauen sind die verschiedenen Klassen bereits vorgegeben. 

Klassen sind vorgegeben, multiclass classification weil CITE WIKIPEDIA
Ziel: Datensatz
Klassen nichts
% Merkmale
Bei der Entwicklung der Vorgehensweise zur Klassifikation der eingegebenen Texte wird davon ausgegangen, dass keine exogene Information zur Verfügung steht. Das heißt, dass der Klassifikationsmechanismus nicht auf Metadaten wie z.B. Autor oder Uhrzeit zugreifen kann. Somit steht nur endogene Information, also der Inhalt der eingegebenen Texte zur Verfügung. Die Prüfung des Vorhandenseins von Merkmalen beschränkt sich also auf Bereich des Textkörpers. \cite{sebastiano}

Beziehung Schema, Merkmale, Klasse
Zur Unterscheidung müssen jeweils die Merkmale der einzelnen Klassen ausgemacht werden. Dieser Vorgang wird in dieser Projektarbeit als Charakterisierung bezeichnet. Zur Charakterisierung wird hier das Schema herangezogen, in dem ein Datensatz der jeweiligen Kategorie gespeichert werden soll. Die Schemata der einzelnen Kategorien definieren jeweils unterschiedliche Felder. 
Jedes Feld hat einen bestimmten Typ bzw. eigene Merkmale.
Die Verteilung von Feldern verschiedener Typen auf die einzelnen Schemata ergeben im Idealfall ein Muster, das nur bei einem einzigen Schema zu beobachten ist.
Tabelle X zeigt, welche Felder die einzelnen Schemata definieren.

\begin{tabular}{lll}
    Klasse & Aufgabe & Servicefall \\
    Schließen \\
    Status \\
    Betreff \\
    Kontakt \\
    Telefon \\
    Fällig am \\
    Mitarbeiter \\
\end{tabular}

nur boolsche Merkmale
problem of notes - bad spelling
Falschpositiv/Falschnegativ
Dieses Muster wird nun zur Charakterisierung der einzelnen Klassen genutzt. Somit ist die Merkmalsverteilung der Klassen analog zur Verteilung der Felder auf den Schemata der einzelnen Kategorien. 
Bei der Betrachtung der Schemata fällt auf, dass die einzelnen Felder immer entweder vorhanden oder nicht vorhanden sind. Da sich die Merkmale analog zu den Feldern der Schemata verhalten, sind erstere als boolsche Merkmale zu bezeichnen. Das heißt, dass sie die Werte 'vorhanden' und 'nicht vorhanden' bzw. 1 oder 0 annehmen können. Nach der Charakterisierung der Klassen werden nun die Möglichkeiten betrachtet, das Vorhandensein von Information, die den einzelnen Feldern entsprechen, und somit die Ausprägung der Merkmale als 'vorhanden ', festzustellen. 
SATZ UMSCHREIBEN
zitat boolsche merkmale

Tabellenspalte > Absoluter Wert, Querverweis, Spaltenname
Bei der Analyse der verschiedenen Schemata fallen auch grundsätzlich verschiedene Informationstypen auf, die die einzelnen Felder erwarten. Die verschiedenen Informationstypen einschließlich der Felder, von denen sie erwartet werden, sind in Tabelle X beschrieben.

\begin{tabular}{lll}
    Informationstyp                 & Felder \\
    Fließtexte                      & Betreff, Beschreibung \\
    Boolsche Werte                  & Status (geschlossen/offen) \\
    Standardformatierte Information & Datum, Telefonnummer \\
    Querverweis                     & Mitarbeiter, Kunde, Kontakt, \\
\end{tabular}

Bei der Klassifikation der Eingabe wird zunächst das Vorhandensein von Informationen

Unabhängig davon wird auch die Möglichkeit in Betracht gezogen, dass der Nutzer selbst bereits die Kategorie des vorgenommenen Eintrags innerhalb jenem vermerkt.



/* Absoluter Wert = Entity / Pattern recognition */
/* Querverweis = String matching */
/* Spaltenname = keyword search */
/* Klasse > Merkmale */
/* Merkmalsraum? */

/* Merkmale > Classifier */
/* Verarbeitung: Eingabe > Merkmale > Konfidenzintervall */
/* art des merkmals */ 
/* Machine Learning */

\section{Extraktion von Information}
Merkmale die vorhanden sind können weiter verwendet werden.
Verstärkte suche
using chrome? for email footer for example
lernprozess - learning when something was missed if user enters it and it is also found inside the text

	\chapter{Einleitung}
\section{Problemstellung}
\section{Motivation}

\chapter{Anforderungen}
\section{Interaktionsmodell}
? Interaktionsmodell
Nutzer / Programme relation

1. Neuen Eintrag
2. Speichern des Eintrags
3. Bestätigung der erfassten Information

Überleitung
\section{Funktionale Anforderungen}
eingaben entgegennehmen
Typ der eingabe erfassen
Informationen extrahieren
Bestätigung der Resultate erfrage, dabei auswahl stellen, weil..
eingabe und erfasste sowie bestätigte daten speichern
lernen
\section{Nichtfunktionale Anforderungen}
Zuverlässigkeit (Systemreife, Wiederherstellbarkeit, Fehlertoleranz)
Aussehen und Handhabung (Look and Feel)
Benutzbarkeit (Verständlichkeit, Erlernbarkeit, Bedienbarkeit)
Leistung und Effizienz (Antwortzeiten, Ressourcenbedarf, Wirtschaftlichkeit)
Betrieb und Umgebungsbedingungen
Wartbarkeit, Änderbarkeit (Analysierbarkeit, Stabilität, Prüfbarkeit, Erweiterbarkeit)
Portierbarkeit und Übertragbarkeit (Anpassbarkeit, Installierbarkeit, Konformität, Austauschbarkeit)
Sicherheitsanforderungen (Vertraulichkeit, Informationssicherheit, Datenintegrität, Verfügbarkeit)
Korrektheit (Ergebnisse fehlerfrei)
Flexibilität (Unterstützung von Standards)
Skalierbarkeit (Änderungen des Problemumfangs bewältigen)
Randbedingungen


\chapter{Theoretische Herangehensweise}
\section{Verarbeitungsprozess}
\subsection{Klassifikation der Eingabe}
\subsection{Extraktion von Information}
\subsection{Bestätigung der Resultate}

\chapter{Implementierung}
\section{Module}
\subsection{Klassifikation der Eingabe}
\subsection{Extraktion von Information}
\subsection{Bestätigung der Resultate}
\section{Validierung}

\chapter{Schluss}


Ziel dieser Projektarbeit ist die Entwicklung eines Systems, dass es dem Nutzer erlaubt, Information ohne Eingabemaske einzupflegen. Dabei soll die Eingabe automatisch, wenn möglich, einer Klasse in der Problemdomäne CRM zugeordnet werden. Wir betrachten Texte, die nur bedingt prosaisch sind und deshalb nur wenig grammatikalische Strukturen aufweisen. In diesem Fall müssen andere Merkmale zur Verarbeitung der Eingabe berücksichtigt werden.
Zu Beginn des ersten Teils der Arbeit werden die konkreten funktionalen Anforderungen an das System definiert. Als Nächstes betrachten wir die theoretischen Prozessschritte, die vom Rohtext zum eingeordneten und extrahierten Datensatz führen. Der zweite Teil behandelt die Schwierigkeiten die in Anbetracht der Qualität der Eingabedaten und der technischen Mittel aufkommen sowie diesen entgegengewirkt werden kann. Im Folgenden werden aus den theoretischen Überlegungen abgeleitete systemarchitektonische Strukturen auf die tatsächliche Umsetzung übertragen.
Zuletzt wird der Erfolg des Systems Anhand der Anforderungen ermittelt und ein Ausblick auf praktische Einsatzmöglichkeiten geboten.

non-tech affinity \cite{hemken}
human indexing is expensive, cost reduction because expert-labour
learning by doing encouraged. people get introduced to the problem domain throught the suggestions served by the system.
Ziel: Klassifikation und Extraktion von Texten
domäne crm, ausgewählte kategorien
\chapter{Hauptteil}
\section{Funktionale Anforderungen}
\subsection{Interaktionsmodell}
Aus der Perspektive des Interfacedesigns werden in diesem Abschnitt die Anforderungen an die Interaktion mit dem Systems definiert.
- ausegehend von liveeinsatz mit echtem nutzer
- vertrauen ist fragil \cite

definition setting,user scenario,interaction model
system erhält feedback von nutzer - trainierbar
interaktionsmodell bzw. lernprozess - learning when something was missed if user enters it and it is also found inside the text
teilautomatische "interaktive" klassifikation und extraction, weil: trainingsdaten rar? und vertrauen auf den classifier gering.
\subsection{Datenverarbeitung}
automatisierte generierung von metadaten, metadaten werden hier das neue zentrum der interessen
exogenous information - phone note for example on ingoing phone call
X verschiedene Datentypen
echte anforderung: email-footer
interaktive Klassifikation bzw. Datenverarbeitung
extensible: insert own types with certain features?
fokus auf texte die einer klasse entsprechen
a priori?
Simon Mayer becomes Simon Mayer (recognised kind: account)
\section{Theoretische Herangehensweise}
Texte fast nie grammatikalisch korrekt
endogene Information
problem of notes - bad spelling
\subsection{Klassifikation der Eingabe}
Im folgenden wird die theoretische Herangehensweise an die Klassifikation der eingegebenen Daten beleuchtet.
Categorization - using chrome? for email footer for example
Occam's Razor?
characterization - finding features of certain category
characteration through template?
\subsection{Extraktion von Information}

\section{Implementierung}
\subsection{Werkzeuge}
alchemy api
learning classifiers
regular expressions

\subsection{Trainingsdatensatz}
- endogenous knowledge only in paper notes...
- nonoverlapping categories but multiple categories in one note. -> separation
problem des trainingsdatensatz: bereits exportierte, verfremdete daten
\subsection{Implementation des Klassifikationsmechanismus}
\subsection{Implementation des Extraktionsmechanismus}
\section{Evaluierung des Systems}
\subsection{Prüfung der funktionalen Anforderungen}
\subsection{Präzision}


\chapter{Schluss}
\section{Zusammenfassung}
\section{Fazit}
\section{Ausblick}

\chapter{Notizen}

section{To Do}
Jones fragen
binäre entscheidung? oder single-label?
find thesaurus for crm
binäre entscheidung? oder single-label?
expert interview - ivan kravchenko?
recognizing special cases (what did i mean here?????)

Datensätze:
E-Mail-Footer
Exporte der CRM-Datensätze

section{Literatur}
Knowledge‐enabled customer relationship management: integrating customer relationship management and knowledge management concepts

"An Adaptive User Interface Based On Personalized Learning" pdf

	Die Spezialisierung auf die Domäne CRBM bedeutet nicht, dass das entwickelte System nur in der Lage ist, die funktionalen Anforderungen nur in jener Domäne zu erfüllen. Im Gegenteil: das System sollte so flexibel sein, dass es auch für Informationssysteme genutzt werden kann, deren Entitäten sowohl an sich als auch untereinander ähnlich strukturiert sind.


	\clearpage
	
	% Ab hier weiter mit großen, römischen Seitenzahlen
	\pagenumbering{Roman}
	
	% Seitenzähler auf zwischengespeicherten Wert für große, römische Zahlen setzen
	\setcounter{page}{\theRomanPagenumber}

	% Kopf links auf Titel ändern
	\ihead[\titel]{\titel}	
	
	% Anhang
	% \addchap{Appendices}
	% \appendix
	% \input{content/Anhang}
	
	% % Beigaben
	% \include{content/Beigaben}
	
	% % Glossar, falls benötigt
	\newacronym{js}{JS}{JavaScript}
\newacronym{ie}{IE}{Informationsextraktion}
\newacronym{ae}{AE}{automatische Entitätserkennung}
\newacronym{crm}{CRM}{Costumer Relationship Management}
\newacronym{b2c}{B2C}{Business to Customer}
\newacronym{b2b}{B2B}{Business to Business}

% Entität
% Objekt
% Token


	\printglossary
	% Literaturverzeichnis
	\printbibliography
	
	% Aktuellen Abschnitt beenden
	\clearpage
\end{document}
